\documentclass[../main.tex]{subfiles}
\graphicspath{{\subfix{../images/}}}
%--------------Begin appendix section:----------------------
\begin{document}
\newpage
\appendixpage
%\appendix
\begin{appendices}
\begin{singlespace}
{\parindent0pt
\section{Contact Information}\label{apxA}
\noindent\textbf{Faculty Advisor(s):}\newline
Steve Betza\\
Professor of Practice in Engineering Leadership\\
The Pennsylvania State University\\
\begin{multicols*}{2}
\noindent\textbf{Student Information}\newline
Michael Galaini\\
Undergraduate Student in Mechanical Engineering\\
The Pennsylvania State University\\
mpg5835@psu.edu
\\~\\
Brendan McLernan\\
Undergraduate Student in Electrical Engineering\\
The Pennsylvania State University\\
bxm5554@psu.edu
\\~\\
Jaden Weed\\
Undergraduate Student in Engineering Science\\
The Pennsylvania State University\\
jrw6156@psu.edu
\columnbreak
\\~\\
Alexander Stepko\\
Undergraduate Student in Electrical Engineering\\
The Pennsylvania State University, Schreyer Honors College\\
axstepko@psu.edu
\\~\\
Sana Tipnis\\
Undergraduate Student in Computer Science\\
The Pennsylvania State University\\
sat5652@psu.edu
\end{multicols*}
}

\section{Description of the University and Engineering College}\label{apxB}
\begin{description}
\item[The Pennsylvania State University]located in State College, Pennsylvnaia and spanning 19 commonwealth campuses across Pennsylvania, has been dedicated to empowering and educating the generation of the future since 1855. In total, the university campuses serve over 80,000 students in varying levels of education from undergraduate to doctoral degrees annually. Penn State prides itself in being a top 25 research university, investing over 1.01 billion dollars in annual research expenditures, and solving problems with profound impacts on our world. The university also takes great recognition for its incredible alumni program, with over 120,000 members in the association, and 700,000 in total living worldwide. \par
\item[The Penn State College of Engineering]is ranked 21st in the country and 12th in the nation in public engineering programs. The College of Engineering's mission statement is, "To prepare engineering graduates to be designers, leaders, and entrepreneurs in all fields of engineering and to disseminate new findings from applied research that makes the world a better place". The College of Engineering values multidisciplinarity, innovation, experiential learning, global engagement, and ethical decision-making. The main objectives of the college is to create a healthy, inclusive, and productive culture and climate in the School, lead high-impact scholarly research that advances interdisciplinary engineering knowledge, practices, and policies, offer innovative and interdisciplinary educational programs and experiences in residence and online that enable engineers to thrive throughout their careers, and broadly communicate and promote our role and impact on society. \par
\end{description}
    
\newpage
\section{Non-University Partner Reserarch Biographies}\label{apxC}
\begin{description}
    \item [Steve Betza] is a former corporate director of the Future Enterprise Initiative at Lockheed Martin. He served as an adviser to the Department of Defense, National Security Agency, and Defense Science Board, helping with electronic, U.S. industrial base, and manufacturing innovation issues. Betza led teams in developing advanced flight computers with IBM and held several executive titles at Lockheed Martin as a key figure in billion-dollar companies. He is also presently a Professor of Practice in Engineering Leadership at The Pennsylvania State University, making him a great candidate to interview for the purposes of that class project. Betza opened our eyes and broadened our horizons for our research and prototyping process. Our team learned most planes used today still have design frames from the '50-'60s, only really being updated internally. This means the option for swapping aircraft internals with alternate propulsion systems is extremely viable. From this, we concluded the transition from the planes we know and fly today to E-planes would be relatively seamless in-plane structure and flight behavior aspects. Our meeting with Betza withheld countless strides of improvement in our research and design process, but perhaps the most important takeaway was that the process of alternating aviation infrastructure into electric-based will be a long and difficult path; but nevertheless, completely attainable and maybe even natural.
    
    \item [Bill Grauer] is a former test engineer for Boeing's wind tunnel, vibration, and propulsion research facilities. He currently consults for Boeing's Philadelphia site and has over 40 years of experience in the aircraft manufacturing industry. Mr. Grauer provided technical insight into aircraft hybridization, biofuels' viability as propulsion energy, and the technical hurdles associated with an electric propulsion system. Being a propulsion test engineer, there was visibility provided into the usage of biofuels in existing or modified engine architecture. Additionally, a brief touch on the usage of rotary-winged aircraft for UAM shows that enormous technical hurdles remain in this sector. Mr. Grauer's involvement in the industry over a broad time period shows there is limited viability in the use of biofuels in existing or new airframes, and that electrically based propulsion is likely to be more well-received by -the airline industry, airport operators, and governmental bodies.
\end{description}

\section{\texttt{Sign-off form for faculty advisor(s) and/or department chair(s)}}
\noindent  This section comprises of a \texttt{.pdf} file that has not yet been filled out by the faculty advisor, in the event that our team decides to submit this publication for the competition.\par

\newpage
\section{Evaluation of Educational Experience Provided by the Project}\label{apxE}
\noindent\textbf{Students}\newline
\begin{enumerate}
    \item Did the Airport Cooperative Research Program (ACRP) University Design Competition for Addressing Airports Needs provide a meaningful learning experience for you? Why or why not?
    \item What challenges did you and/or your team encounter in undertaking the competition? How did you overcome them?
    \item Describe the process you or your team used for developing your hypothesis.
    \item Was participation by industry in the project appropriate, meaningful and useful? Why or why not?
    \item What did you learn? Did this project help you with skills and knowledge you need to be successful for entry in the workforce or to pursue further study? Why or why not?
\end{enumerate}
\noindent\textbf{Faculty}\newline
\begin{enumerate}
    \item Describe the value of the educational experience for your student(s) participating in this competition submission
    \item Was the learning experience appropriate to the course level or context in which the competition was undertaken?
    \item What challenges did the students face and overcome?
    \item Would you use this competition as an educational vehicle in the future? Why or why not?
    \item Are there changes to the competition that you would suggest for further years?
\end{enumerate}

\newpage
\section{References}\label{apxF}
\printbibliography[heading = none]
\end{singlespace}
\end{appendices}
\end{document}