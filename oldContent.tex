



%Use this area for content that is no longer used in the final publication!

Implementing large scale energy storage and power generation systems would be key in making airports greener, as they would no longer rely on energy created by fossil fuels. It is not only a greener solution than the status quo, but when your account for the future implementation of electric planes, power storage becomes an absolute necessity. In the best-case scenario, airports implement power generation and critical future charging infrastructure for electric planes. However, even in the worst case scenario, where electric planes do not become practical for several decades, or even at all, this solution still manages to make airports far greener. In the worst case scenario, there still exists essentially a small power plant at the airport, as well as energy storage. The energy storage alone would allow for a less volatile swing in energy generation and cost for the surrounding local community, by keeping the energy in the grid more easily accessible. Depending on the level of energy generation at an airport, you may even be able to power the airport in full and even sell of the extra energy to the grid as well. \par
One of the key challenges with any large-scale project, such as on-site power generation and storage, is trying to convince the custom that the investment is worth their while. Customers often decide this buy evaluating cost, risk, and potential benefit of the product. Different customer value these three elements differently, one the most unique customers are governments. Most airports are owned in part or in full by local, state, or federal governments. Governments are an ideal customer for the implementation for on site power generation and storage at airports because of its huge initial cost, yet massive benefit it would have for the surrounding community. Adding these elements to airports are an investment that even in the worst case scenario benefits the local community and eventually become economically viable, and in the best case largely benefits the community financially and environmentally.  \par
