\documentclass[../main.tex]{subfiles}
\graphicspath{{\subfix{../images/}}}

\begin{document}
\section{Literature Review}\label{litReview}
\subsection{Introduction}\label{designIntro}
The airline industry’s heavy reliance on a petroleum-based operational model poses significant challenges when trying to transition towards a sustainable, environmentally conscious, and yet still scalable business model. Currently, aircraft and aerospace industry manufacturers see two viable solutions for environmentally friendly air travel: electrically powered aircraft, and biofueled aircraft. The implementation of both solutions in the airline industry poses significant challenges; however, research explored in this publication demonstrates that these challenges can be overcome in both the electric and biofuel segments of sustainable air travel.\par
\subsection{The Impact of Electric Aircraft on Future Markets}
It is reasonable to expect that the transition towards electrically-powered planes will happen gradually, as is the case with many changes in air travel. There are significant barriers to implementing the electrification of an airport, namely a high initial investment period and long clearing time before profit is again seen (
4).  Additionally, governmental oversight into the safety validation and operational stability of new electric systems further adds to transition timeframes.\cite{ref24} This may discourage investment in Global Electric Aircraft Markets, yet a report conducted by Vantage Market Research shows favorable predictions. The 2021 valuation of this aircraft market was \$7.7b USD; in 2028, this valuation increases to \$17.8b USD. Other reports show potential market increase up to \$27.7b USD in the same year (2028) (REF 26). Predictions for widespread adoption of EA in the passenger transport / UAM sector would likely be between 2025 and 2035, given multiple variables including (but not limited to): market profiles, power infrastructure installations, utility support, permitting and approval processes, and construction times (REF 40).\par
This large increase in market share is emphasized in Asia-Pacific regions, however there is also substantial market growth available in the logistics industry (REF24). The advent of eCommerce shows that consumers demand products as fast as possible, regardless of the means by which it must arrive. The usage of smaller, electric planes for logistics has proven itself to be cost effective and efficient for parcel carriers (REF 27). DHL is an industry leader in logistics electrification, with hopes to create the “…world’s first electric air transport network.” (REF 27).\par

\subsection{Target Markets of Electric Aircraft and its Related Infrastructure}
There also exists potential demand in the urban air mobility (UAM) sector (REF 26), coinciding nicely with limitations related to power density of thoroughly-tested battery technology (REF 25 ). Aircraft used for this sector would likely be business jets, regional transport aircraft, and ultralight aircraft, as the smaller size of these machines provide a more favorable stance for electrification (REF 27). Private investors will see EA as more cost effective, as electric motor have proven to be more mechanically and financially efficient (REF 27).\par
The North American market accounts for the largest share of the EA industry at 34.3\% (REF 26). Favorably, the expected CAGR during the forecast period is 16.1\%. With these favorable market predictions, American Airlines has recently invested \$25m USD into Vertical Aerospace Group with the hopes of gaining penetration into UAM networks.  A major hurdle to overcome in electrically powered aircraft is energy storage methods. Current battery technology is capable of power densities in the realm of 250  Wh/kg. Typical jet fuel has a power density of ~12000 Wh/kg (REF 25). For comparative energy storage, an EA with the same capacity would weigh approximately 30 times more than its hydrocarbon counterpart (REF 26).  Advancements in battery technology from Lithium-polymer to Lithium-sulfur based chemistry may yield power density increases from 250 Wh/kg to ~500 Wh/kg , however this still pales in comparison to hydrocarbon-based fuels – the major hurdle to mass-electrification of an air transport network (REF 26). Consequently, current market predictions focus on logistic and UAM developments.\par
\subsection{Small-Scale Deployments of Electric Aircraft Infrstructure}
A crucial aspect of any airport operational model is energy transfer. Traditionally, this is accomplished via refueling with hydrocarbon fuels. Electric aircraft, obviously, use no hydrocarbons for their main propulsion system. Instead, the development of battery charging areas at the airport is necessary. As is expected of major system deployments, it is reasonable to expect EA infrastructure deployed on smaller air travel networks before implementation at large commercial sites.\par
Although not directly related to the topic of EA, the electrification of other airport support vehicles such as pedestrian busses and Ground Support Equipment (GSE) appears to have reasonably small barriers to deployment. The usage of electric buses in an urban setting has already shown success in both maintaining customer throughput rates and reducing the network’s impact on the environment.  It is reasonable then to assume that implementing this infrastructure into an airport would hold similar results (REF 26). Complicated power management algorithms can be developed to maximize charging time under minimal electricity costs, which is generally during non-peak passenger throughput times (REF 26).\par
Manufacturers of EA infrastructure have already made great progress in small-scale deployments. US company Eaton’s Skycharge charging system claims great improvements to the charging experience of an EA. Among other improvements,  Eaton claims to have developed a solution capable of deployment to “any e-airplane, EVTOL [Electric Vertical Takeoff and Landing], or UAM aircraft (REF 3). This inter-operability of multiple aircraft charging standards provides a large amount of flexibility to airports without having to consider the implementation of multiple charging standards. Being environmentally conscious, Skycharge’s system architecture also has framework to implement Eaton photovoltaic (PV) systems directly into the charger with ease (REF 3). The geographical requirements of an airport are such that a large, flat, open space is needed. Airports can be outfitted with PV systems (designed to eliminate pilot glare and obstruction), and then act as energy storage/distribution centers to not only their own infrastructure needs, but also to the grid (REF 26).\par
Smaller General Aviation airports have already begun to implement EA charging into their operational model. The Compton/Woodley Airport (FAA identifier CPM) in Los Angeles County, California has already integrated such a system into its existing energy-delivery network (REF 28). There has also been developments in Sweden for implanting charging infrastructure for fossil-free aviation. They set up three total charging stations, where two of them were set on the main apron of the airport and the other located in the southern part of the airport. This ensured aircrafts currently on the market the ability to utilize these charging stations, however, the duration of these potential flight times are unknown. The Bern-Belp airport in Switzerland utilizes the Eaton Skycharge system with good success (REF 3). Martha’s Vineyard Airport (FAA Identifier MVY) has also begun to investigate the implementation of EA charging into its operational model working alongside Beta Aircraft and Aviaton Technologies (REF 29). Sam Hobbs, of Beta, notes that chargers for their planes also work with, “…Teslas, Chevy Volts, and other electric vehicles.” (REF 29), demonstrating the flexible energy demands of light aircraft. Beta’s charging network is expansive, spanning much of the eastern seaboard and southeast United States (REF 30). The target market of this manufacturer is logistics and UAM, suggesting that reports produced by Vantage Markets Research \cite{ref24} and MarketsandMarkets Research Private Ltd. (REF 26) are accurate in their predictions. Similarly to Skycharge, their charging system emphasizes flexibility and ease-of-integration into existing power distribution channels (REF 30). For example, commercial aviation (including but not limited to scheduled, air taxi, and charter operations) fly with strict timetables. These aircraft require fast, on-demand charging likely during peak grid usage hours. General aviation is not subsequent to such strict timing demands, and therefore have more relaxed charging demands. These aircraft would be best to charge at non-peak electricity demand time periods. Regardless of the category of operation, demand management systems would be vital to ensure efficient operation (REF 40).\par
Of course, charging on-board batteries for EA is not the only potential solution. To greatly decrease turnaround time for an airframe, some concept designs feature capability to swap entire battery modules between planes and charging bays, eliminating the airframe’s downtime to charge batteries (REF 31). In addition to time savings, out-of-airframe charging with battery swaps ensures adequate cooling, charge rates, and the subsequent preservation the battery chemistry (REF 41). Amidst the great potential time and resource savings that swap-based infrastructure holds, it is unfortunately clear that the implementation of such a system is generally not feasable in many applications. A complicated analysis of a single airlne utilizing a single battery system shows major operational challenges when trying to manage prompt departures, arrivals, and battery management (REF 31). The conclusion of this research is that even at this incredibly small scale, there are enormous operational obstacles and inefficiencies that exist with the usage of battery swaps (REF 31). With the advent of new fast-charging batteries, the existing barriers with swap strategy become slightly more reasonable; however there still exists inherent advantages in utilizing on-airframe charging.  The concept of utilizing a swap strategy is tantalizing, yet currently remains unreasonable for the foreseeable future (REF 32).\par
\end{document}