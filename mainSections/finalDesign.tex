\documentclass[../main.tex]{subfiles}
%\graphicspath{{\subfix{../images/}}}

\begin{document}
\section{Problem Solving Approach and System Design}\label{finalDesign}

Prior research into the topic of alternatively powered aircraft has shown great viability in the area of electric aircraft, especially within the areas of short-haul cargo logistics. Right-on-time supply chain architecture, especially in areas of e-Commerce has created a demand for quick, efficient package distribution networks. As a result, major logistic carriers see electric aircraft as a viable option to satisfy both the consumer and their sustainability goals.It is also evidently clear that both package carriers and airports do not retain internal resources capable of designing such an infrastructure system for electric aircraft that maintains consistency and viability during the lifetime of the system.\par
Our solution acts as the consultant for both airports and/or logistics companies as a system integrator for this problem. In order for such a solution involving electric aircraft to be environmentally sound, it is crucial to consider the complete infrastructure system necessary to support the new form of air transport. These system elements can be broadly comprised into the following categories:
\begin{singlespace}
\begin{enumerate}
    \item Marketing of Electric Aircraft Integration
% * <Alex Stepko> 17:23:01 31 Jul 2022 UTC-0400:
% This is an awkward title, not sure what to put here.
    \item Power Generation
    \item Power Storage
    \item Power Distribution
    \item Energy Transfer to Aircraft
\end{enumerate}
\end{singlespace}

\subsection{Marketing of Electric Aircraft Integration} % MICHAEL'S SECTION
The adoption of Electric planes in the flight industry is not a question of if or how, but when. At this point in time, the addition of electric airplanes to our skies is merely a natural progression we are now witnessing and will continue to witness. This new market brings a better today along with a cleaner tomorrow by combining cheaper flights with greener transportation. In combination with a sustainable power generation source such as anti-glare solar panels, moderately frequent flights will cost virtually nothing in “fuel” or charging rates. Creating green power generation will pay for itself over and over as airports continue to use this energy for their flights creating cost efficient flight affiliated fares for both the passengers and airports.\par
When discussing just the cargo and shipping capabilities, electric planes will dominate the industry. The world is constantly attracted to “faster, better, smarter” technology, and the biggest demand in the shipping world is speed. Overnight shipping, two-to-three-day shipping, you name it, every large bountiful product transport company has it. Electric planes, especially VTOL models, are ideal for not only assuring these fast and demanded services but creating a cleaner wave of transportation in the process. The most accepted form of electric aircraft are currently smaller airplanes ideal for frequent fast flights such as overnight shipment delivery.\par
A large benefit of green airfare is that in the event of a fuel shortage or large spike in jet fuel, electric planes will be barely affected. Because most of the “refueling” or charging process, if not all is done naturally, electric airplanes will flourish in the event of a liquid fuel shortage or inflation. Airlines can offer exclusive deals, rates, and opportunities unheard of to customers through the electric plane avenue, changing the dynamic of flight fares.  The earlier an airport accepts and inducts these aircrafts, the more profitable the airport will be in the near future.\par 

\subsection{Sustainable Power Generation} % ALEX'S SECTION
The concept of sustainable power generation is an inherently complex topic that in of itself can constitute a solution. However, our research has shown there is great value in providing a solution capable of performing the analysis necessary to implement a sustainable power generation system at a given site. A variety of factors contribute to the overall design, including but not limited to: climate, geography, flight throughput, maintenance costs, and the life cycle of the system. It is crucial to have a rudimentary understanding of these concepts in order to properly design the system.\par
Climate analysis is the first aspect of a power generation system. Sustainable solutions require climate that is conducive to the generational technique utilized. For example, solar generation requires climate with consistent solar radiation. Similarly, low-level solar turbines require consistent wind and weather patterns. In coastal areas, tidal generation systems may be used. However, an oceanographic survey is necessary to ensure favorable tidal patterns.\par
Once a basis for the site climate characteristics is established, it is next necessary to examine the geography available at the site. Fortunately, many airports feature large and somewhat flat land. This is primarily for safety and also gives incoming pilots an easy approach path. Sites that have a more complex terrestrial makeup constitute a more unique solution, presenting a unique set of challenges regardless of the power generation method utilized.\par
Most crucially, the system design must contain enough generation capacity. Without sufficient capacity, a sustainable solution is helpful, albeit ineffective at providing green energy to the aircraft. At existing sites, it is simple enough to perform an assessment of the cargo throughput, planes utilized, and operational schedule. These metrics can then be applied to the target electric fleet planned at the site, which results in the amount of power necessary to support that given capacity. It is advantageous to carefully consider the growth trajectory of the site when looking at throughput, as traditional cumulative growth rates may not be entirely accurate in multi-decade time frames.\par
The aforementioned assessments culminate into a site model that can be utilized in the selection of various power generation methods.

\subsection{Power Storage} % BRENDAN'S SECTION
The power storage aspect of our solution is complementary to the sustainable generation of power.  There are multiple options for storing and/or offloading the generated power that an airport must consider. 
The first option is investing in large scale battery storage so that the generated electricity can be stored before an electric aircraft requires recharging. This option would be favorable for an airport that utilizes battery-swapping methods for aircraft rather than recharging-- a concept that is further detailed in the "Energy Transfer to Aircraft" section. However, this can be a costly consideration that loses efficiency as the energy is transferred between batteries. The battery technology is also still improving, so such an investment would be ineffective in the long-term. 
Alternatively, an airport could use its associated electrical grid as somewhat of a battery. The airport would offload it's  electricity into the grid as it is generated, and when aircraft require charging they can pull from this same grid. This option would be less viable in remote areas considering the electrical grid would have lower demand. This is favorable near large cities or other areas with infrastructure demanding large amounts of electricity (i.e. refineries, factories, etc.). This is also a relatively low-risk option for an airport because the ability to offload the generated energy would be sustainable in the unlikely event of electric aircraft failing. 

\subsection{Power Distribution} % ALEX'S SECTION

\subsubsection{Energy Transfer to Aircraft} % SANA'S SECTION
Once we have decided where we store this power and how we actually distribute this power, it is paramount to consider how we will transfer this energy to their aircraft using our design model. We know that traditionally, this is done by refueling with hydrocarbon fuels through a complicated network of tanks and pumps. Maintenance of the fuel distribution system is extremely expensive for an airport. With the usage of electric aircraft, airports no longer need to consider the logistics of fuel transfer. Instead, energy transfer is accomplished via battery chargers.\par
Battery charging can be split into two pathways, swapping batteries and recharging batteries. Depending on specific facets of the airport, airports could adopt either model to best benefit their aircraft. First, we would consider swapping batteries. To greatly decrease turnaround time for an air frame, we would feature the capability to swap entire battery modules between planes and charging bays, eliminating the air frame’s downtime to charge batteries. This is a much more efficient solution, and the preferred one, ir airports have the capacity to do so. Out-of-airframe charging with battery swaps ensures adequate cooling, charge rates, and the subsequent preservation the battery chemistry  
The other model is recharging batteries, likely the more feasible solution for airports, especially during the beginning stages of electric aircraft implementation. This would require charging stations to be set up in the hangers in the aircraft. A scaleable model would be built to implement these charging stations depending on the size of the aircraft. This way, aircraft could have a set road map to follow of how to implement these charging stations, no matter the size, and ensure uniform implementation across the board. 
We have thought about the different ways energy can be transferred to the aircraft, and through both swapping and recharging batteries, aircraft are able to make sure their planes get the proper charge and care they need to provide energy to their trips. 

 including a description of processes that would need to be undertaken to bring the design to the product/ implementation state. Emphasis should be on increased affordability and utility.
% * <sat5652@psu.edu> 18:06:47 04 Aug 2022 UTC-0400:
% still need to hone in on this in my section...will do tonight

This part of the solution would meet ACRP goals, specifically goal 3 which states, "Engage students at U.S. colleges and universities in the conceptualization of applications, systems and equipment capable of addressing related challenges in a robust, reliable and comprehensive manner." By thoroughly analyzing the different ways energy can be transferred to these airports, because we know not every airport can have a one size fits all solution, we have thoroughly looked at the challenge in a comprehensive manner, bringing a detailed final prototype to help combat this issue. 
The projected benefits of this solution would be the feasibility for any airport to implement this issue to ensure clean power generation. 

\section{Safety Risk Analysis} % JADEN'S SECTION

\section{Cost Benefit Analysis} % ALEX'S SECTION

The broad and applicative nature of our solution requires a site selected for analysis. For simplicity and convenience, the University Park Airport located in State College, Pennsylvania was chosen. This airport features stable passenger and cargo throughput, and an easily analyzed operational structure. Basic statistics on the airport are also available through a \emph{Sustainable Master Plan} published by the airport itself \cite{SCEplan} in 2016, which will suffice for a demonstration of the solution.\par
% * <Alex Stepko> 23:24:38 03 Aug 2022 UTC-0400:
% this might get deleted, not sure yet.
% * <Alex Stepko> 22:36:41 03 Aug 2022 UTC-0400:
% We may want to move this to the end of the paper as an intro to the CBA.


As mentioned in section \ref{sysOverview}, placing quantitative values on our solution's viability requires a specific deployment site. This then provides the necessary data required for analysis of a certain part of the system. The University Park Airport has made available their \emph{Sustainable Master Plan}, which gives basic statistics pertaining to cargo throughput \cite{SCEplan}. With these statistics in mind, a Cost-Benefit-Analysis was developed, and can be seen below in Figure \ref{fig:CBA}.


\begin{figure}
\centering
\includegraphics[width = 7.5in, angle = -90]{CBAv1.pdf}
\caption{University Park Airport Cost Benefit Analysis}
\label{fig:CBA}
\end{figure}
\FloatBarrier

\end{document}