\documentclass[../main.tex]{subfiles}
\graphicspath{{\subfix{../images/}}}

\begin{document}
\section{Problem Statement and Background}\label{probStatement}

Current aviation infrastructure centers itself around a petroleum-based operational model. This operational model, developed over decades by governmental agencies, airport authorities, and airlines has developed into a multi-billion-dollar industry responsible for the safe transportation of almost 3-million individuals every day in the United States alone \cite{ref19}. Despite the massive success seen in the airline industry and the overall increase in accessibility of air travel to the average citizen, there is now a major push by global governing bodies towards a sustainable future. Currently, the fuel used by jet aircraft produce various greenhouse gases that pollute more than just CO2, namely pollutants such as Nitrogen Oxides, Sulphur Dioxide, soots, and various other microparticles \cite{ref11}. These 'non-CO2' pollutants, in addition to the CO2 emissions inherent in a hydrocarbon-based fuel, pose an adverse health risk to people in contact with it and the area surrounding the airport. Additionally, there is substantial evidence that exposure to common fuels such as Jet A and Jet A-1 likely increases people's susceptibility to respiratory infections, asthma, and other chronic lung infections \cite{ref12}\cite{ref20}. In addition to health risks, fuel spills pose an enormous environmental impact, and the chemistry and behavior of the fuel itself when released into the environment make it difficult to contain \cite{ref14}\cite{ref15}.\par 
In addition to the adverse environmental effects that jet fuels pose, there are some operational issues to consider with our current aircraft support infrastructure. Commonly used fuels are kerosene-based, which is currently a derivative from fossil fuels. Due to this reliability on a single base hydrocarbon, supply chain and market instability has led to fuel shortages and price hikes. The result of this is an unstable industry revenue stream, whose burden is placed on the consumer \cite{ref13}. In addition, electrical charging infrastructure is not widely in place and is much more costly to install compared to the existing fuel infrastructure \cite{ref2}. The airline industry lacks penetration into affordable short-haul flights, which has immense potential for the future, but since CO2 emissions are not at breakeven, this remains widely unattainable using current methods. Current battery technology, although at a great disadvantage in power density compared to current fuel chemistry (W/Kg), provides a platform to operate short-haul aircraft on a smaller scale.\par 
The environmental and operational impacts currently seen with the usage of traditional jet fuels, although successful in past decades, have demonstrated that there is a desire for change in how we provide energy to planes. Alternative energy sources are critical to power the planes of the future. Improving the materials and technologies that engines use today could lead to 70\% more fuel-efficient aircrafts and engines compared to 40 years ago \cite{ref21}. The benefits of using sustainable aviation fuel (SAF) also extend beyond the environmental benefits. The usage of biomass-based fuel leads to improved aircraft performance and additional revenue for farms where the biomass was obtained \cite{ref18}. Some of the sustainable alternative fuels currently in development show that these fuels can be 80\% less carbon-intensive  than traditional fossil-based jet fuels \cite{ref22}. As a result, alternative fuel sources are inherently safer for the people and the environment and have the potential to make airport operations much more seamless and progressive. \par 
Despite the previous operational success seen with the traditional hydrocarbon-based energy model, the change to SAFs and Electric Aircraft (EA) changes must be made soon, as the environmental impacts are creating an unsustainable future for our planet. Not only do these changes affect the environment, but they pose a risk to outdated airport systems. Antiquated airports are unable to handle these new advancements in people and technology without substantial modification to the current operational model. The International Air Transport Association (IATA) reports a 123.2\% increase in jet fuel prices over the past 12 months (as of June 3, 2022) \cite{ref7}. On commercial airliners, jet fuel accounts for about 40\% of the entire cost \cite{ref23}.  It is extremely costly to run these aircrafts and engines the way they are currently designed, which poses a major problem.   With the major increase in discussion about our environmental impact on the planet, aircraft manufacturers have already begun the shift towards using SAFs in existing engines. This reduces the environmental burden slightly, but chemists and engineers are well on their way toward a completely sustainable future.  Airports must do the same and be equipped to deal with this change in energy delivery. As mentioned earlier, Electric Aircraft (EA), an extremely plausible alternative energy aircraft, are becoming more popular. However, airports en masse currently lack the infrastructure to support them on a large flight network \cite{ref9}. Our airports need to be equipped to move forward and handle these demands of modern technology. Therefore, it is imperative we start looking at how to make a change soon, to protect not only our planet but also, to make our airports more progressive to keep up with the ever-changing demands of a technologically advancing society.\par  
The potential short-term impact of making these changes is largely a reduction in CO2 emissions. Airplanes make up 3\% of CO2 emissions but are the fastest-growing source of it \cite{ref6}.  There is an opportunity to grow the labor force as well. An additional set of labor is needed if multiple fuels are in wide use and more electricians will be needed on-site to troubleshoot issues with the widespread electric infrastructure necessary for EAs. Fire suppression techniques also will need to adapt to battery technology. The NFPA's standard 418 is working on techniques for this new sector \cite{ref17}, so we will need a labor force to tackle these issues in both training and equipment. An airport that is prepared for a wide variety of sustainable aircraft will find itself nimbler during this transitionary period and will likely see greater investment from airlines. The result of this is plane manufacturers themselves have an incentive to invest, due to larger numbers of plane purchases using new technology \cite{ref6}.\par 
It is also imperative to consider the transition to these future solutions. To accommodate an electric aircraft, the charging structure would need to be implementable at all airports as well as a means of producing most of the electricity to charge them \cite{ref4}. The larger airports (~30 sites in the US) can utilize large solar arrays, but this will prove more difficult for the sma ller ones (5050 in US) \cite{ref4}. Large infrastructure, such as biofuel-specific pipelines may be necessary to ensure carbon neutrality throughout all aspects of the fuel's life \cite{ref5}. Fueling, although chemically different, will remain similar, but the volatility of new fuels may require mor e precise control of chemistry in storage -- another potential hurdle airports must prepare for. There are several considerations to explore, and the transition will need to be feasible for airports across the U.S. to complete, so looking at the current obstacles in place and actively considering them in design is crucial. We expect planes to be retrofit to accommodate new fuels as well, making this transition much more feasible.\par
Our project aims to create new horizons for airports to easily adjust, encourage, and respond to future alternative powered aircrafts. We need to find a way for airports to create clean energy and cater to these alternate powered aircrafts, without hurting the environment consequently. We are aiming to create not only a solution for big public airports, but for the private sector as well.
\end{document}